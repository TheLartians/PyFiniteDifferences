

With equation \eqref{eq:fresnel_convolution} we derived an exact update rule for $u_{i,j}^{k} \rightarrow u_{i,j}^{k+1}$ when $F$ is constant in the interval $[z_k,z_{k+1})$. To derive an efficient algorithm we first write the convolution as a multiplication in Fourier space using the 2D Fourier transform $\mathscr{F}$ in $x$ and $y$ direction~\cite{wiki:convolution_theorem}

\begin{align} \label{eq:fresnel_fourier_convolution}
\begin{split}
u(x,y,z+\Delta z) &= u(x,y,z) * \mathcal{U}(x,y,\Delta z) \\
&= \mathscr{F}^{-1}[ \mathscr{F}[u(x,y,z)] \cdot \mathscr{F}[\mathcal{U}(x,y,\Delta z)] ].
\end{split}
\end{align}

Since $A$ is imaginary (or more importantly: $\Re(A) = 0$), the Fourier transform of $\mathcal{U}(x,y,\Delta z)$ is given by:

\begin{align*}
\mathscr{F}[\mathcal{U}(x,y,\Delta z)](k_x,k_y) = \frac{1}{2 \pi} \exp \! \left( F \Delta z - A \Delta z (k_x^2+k_y^2)\right).
\end{align*}

If we interpret the discretized field as a Dicac comb and assume periodic boundary conditions, the Fourier transform in equation \eqref{eq:fresnel_fourier_convolution} becomes the the 2D Discrete Fourier Transform (DFT) and its inverse (iDFT)~\cite{wiki:DFT}. This gives us an update rule for the field matrix $(u^k)_{i,j} = u^k_{i,j}$

\begin{align} \label{eq:fresnel_DFT}
u^{k+1} = \exp(F \Delta z)  \, \mathrm{iDFT} \! \left[ \mathrm{DFT} \! \left[ u^{k} \right]\!(k_x,k_y) \cdot \exp \! \left(-A \Delta z (k_x^2+k_y^2)\right) \right],
\end{align}

where the multiplications are performed element-wise with the discretized field. By calculating the DFT using a Fast Fourier Transform this step can be performed in $\mathcal{O}(n_x n_y \cdot log(n_x n_y))$ time complexity~\cite{wiki:FFT}.

Note that this formulation of the Fresnel propagator can also give an approximate solution for non constant $F$ when the step size $\Delta z$ is chosen small enough that diffraction effects in $x,y$ direction between neighboring cells are negligibly small\todo{proof if possible. Reference to multislice-paper?}.














